\documentclass[]{article}
\usepackage{lmodern}
\usepackage{amssymb,amsmath}
\usepackage{ifxetex,ifluatex}
\usepackage{fixltx2e} % provides \textsubscript
\ifnum 0\ifxetex 1\fi\ifluatex 1\fi=0 % if pdftex
  \usepackage[T1]{fontenc}
  \usepackage[utf8]{inputenc}
\else % if luatex or xelatex
  \ifxetex
    \usepackage{mathspec}
  \else
    \usepackage{fontspec}
  \fi
  \defaultfontfeatures{Ligatures=TeX,Scale=MatchLowercase}
\fi
% use upquote if available, for straight quotes in verbatim environments
\IfFileExists{upquote.sty}{\usepackage{upquote}}{}
% use microtype if available
\IfFileExists{microtype.sty}{%
\usepackage[]{microtype}
\UseMicrotypeSet[protrusion]{basicmath} % disable protrusion for tt fonts
}{}
\PassOptionsToPackage{hyphens}{url} % url is loaded by hyperref
\usepackage[unicode=true]{hyperref}
\hypersetup{
            pdftitle={Survival HW3},
            pdfauthor={David},
            pdfborder={0 0 0},
            breaklinks=true}
\urlstyle{same}  % don't use monospace font for urls
\usepackage[margin=1in]{geometry}
\usepackage{graphicx,grffile}
\makeatletter
\def\maxwidth{\ifdim\Gin@nat@width>\linewidth\linewidth\else\Gin@nat@width\fi}
\def\maxheight{\ifdim\Gin@nat@height>\textheight\textheight\else\Gin@nat@height\fi}
\makeatother
% Scale images if necessary, so that they will not overflow the page
% margins by default, and it is still possible to overwrite the defaults
% using explicit options in \includegraphics[width, height, ...]{}
\setkeys{Gin}{width=\maxwidth,height=\maxheight,keepaspectratio}
\IfFileExists{parskip.sty}{%
\usepackage{parskip}
}{% else
\setlength{\parindent}{0pt}
\setlength{\parskip}{6pt plus 2pt minus 1pt}
}
\setlength{\emergencystretch}{3em}  % prevent overfull lines
\providecommand{\tightlist}{%
  \setlength{\itemsep}{0pt}\setlength{\parskip}{0pt}}
\setcounter{secnumdepth}{0}
% Redefines (sub)paragraphs to behave more like sections
\ifx\paragraph\undefined\else
\let\oldparagraph\paragraph
\renewcommand{\paragraph}[1]{\oldparagraph{#1}\mbox{}}
\fi
\ifx\subparagraph\undefined\else
\let\oldsubparagraph\subparagraph
\renewcommand{\subparagraph}[1]{\oldsubparagraph{#1}\mbox{}}
\fi

% set default figure placement to htbp
\makeatletter
\def\fps@figure{htbp}
\makeatother


\title{Survival HW3}
\author{David}
\date{Due February 26, 2020}

\begin{document}
\maketitle

\subsubsection{Question 1}\label{question-1}

\begin{enumerate}
\def\labelenumi{(\alph{enumi})}
\tightlist
\item
  There is a significant difference in relapse associated with
  dactinomycin use (p=0.0439). The hazard ratio for relapse comparing
  those who receive dactinomycin to those who do not is 0.738 (95\% CI:
  0.5550-0.992).
\end{enumerate}

\begin{enumerate}
\def\labelenumi{(\alph{enumi})}
\setcounter{enumi}{1}
\tightlist
\item
  There is a significant association between treatment and relapse after
  controlling for white blood cell count and age using a Cox PH model
  (p=0.0492). The hazard ratio for relapse comparing those that received
  dactinomycin to those that did not is 0.744 (95\% CI: 0.553-1.000).
\end{enumerate}

\begin{enumerate}
\def\labelenumi{(\alph{enumi})}
\setcounter{enumi}{2}
\tightlist
\item
  For the sub-population in which the white blood cell (wbc) count is
  below 10,000, the HR comparing individuals who receive treatment as
  compared to those who did not is 0.553 (95\% CI: 0.368-0.831). For the
  sub-population in which the wbc count is greater than or equal to
  10,000, the HR comparing those who received treatment to those who did
  not is 1.016 (95\% CI: 0.699-1.477).
\end{enumerate}

\begin{enumerate}
\def\labelenumi{(\alph{enumi})}
\setcounter{enumi}{3}
\item
  The HR comparing those who received treatment to those who did not,
  adjusting for wbc, age, and treatment site is 0.710 (95\% CI:
  0.520-0.969).
\item
\end{enumerate}

\includegraphics{Coomes_HW3_files/figure-latex/plot_curves-1.pdf}

\newpage

\paragraph{Question 2}\label{question-2}

\begin{enumerate}
\def\labelenumi{(\alph{enumi})}
\item
  The HR of exit from maintenance comparing individuals who receive a
  one mg higher dose of methadone, adjusting for incarceration status
  and clinic, is 0.965 (95\% CI: 0.953-0.977). The HR comparing those
  who were incarcerated to those who were not, adjusting for methadone
  dose and clinic, is 1.386 (95\% CI: 0.999-1.924). The HR comparing
  individuals from clinic 2 as compared to clinic 1, adjusting for
  methadone dose and incarceration status, is 0.364 (95\% CI:
  0.239-0.555).
\item
  The HR of exit from maintenance associated with an one mg increase in
  methadone dose, adjusting for incarceration history and clinic, is
  0.966 (95\% CI: 0.953-0.978). The HR comparing those that have a
  previous history of incarceration to those that do not, adjusting for
  methadone dose and clinic, is 1.476 (95\% CI: 1.060-2.056).
\end{enumerate}

Using clinic as a stratifying variable allows for a more flexible model
- we are not forcing the baseline hazard function to be proportional
across levels of the stratifying variable (clinic in this case). This is
useful since we don't necessarily care about the HR comparing across
levels of clinic. This does not change the interpretation of our
estimates. It did not change our estimates much, except that our HR
comparing incarceration status is now significant at the 0.05 level
where it had been just barely not significant in the non-stratified
model.

\begin{enumerate}
\def\labelenumi{(\alph{enumi})}
\setcounter{enumi}{2}
\item
\end{enumerate}

\begin{verbatim}
## Call:
## coxph(formula = surv.adix ~ dose + strata(clinic) + dose * prison, 
##     data = adix)
## 
##   n= 238, number of events= 150 
## 
##                  coef exp(coef)  se(coef)      z Pr(>|z|)    
## dose        -0.038981  0.961769  0.008211 -4.747 2.06e-06 ***
## prison      -0.193255  0.824272  0.779144 -0.248    0.804    
## dose:prison  0.009956  1.010006  0.012968  0.768    0.443    
## ---
## Signif. codes:  0 '***' 0.001 '**' 0.01 '*' 0.05 '.' 0.1 ' ' 1
## 
##             exp(coef) exp(-coef) lower .95 upper .95
## dose           0.9618     1.0398    0.9464    0.9774
## prison         0.8243     1.2132    0.1790    3.7956
## dose:prison    1.0100     0.9901    0.9847    1.0360
## 
## Concordance= 0.651  (se = 0.026 )
## Likelihood ratio test= 34.5  on 3 df,   p=2e-07
## Wald test            = 32.23  on 3 df,   p=5e-07
## Score (logrank) test = 33.34  on 3 df,   p=3e-07
\end{verbatim}

\end{document}
